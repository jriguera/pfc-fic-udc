\chapter[Plan de Proyecto]{
  \label{chp:plan}
  Plan de Proyecto
}
\minitoc
\newpage

Como paso previo a la realización de un proyecto es necesario realizar una serie de
análisis que -dependiendo del tipo de sistema a desarrollar- determinen su
viabilidad y permitan intentar predecir los problemas y los riesgos que se
encontrarán durante su elaboración. Por ejemplo, en un proyecto de ingeniería,
si los problemas de desarrollo superan a las expectativas vinculadas a las
predicciones y previsiones, no compensaría el esfuerzo de implementación.
En este capítulo, se estudia el plan de proyecto para la propuesta realizada,
sin tener en cuenta las futuras ampliaciones.


\section{Análisis de Viabilidad}

Al emprender el desarrollo de un proyecto los recursos y el tiempo deben ser
realistas para su materialización a fin de evitar pérdidas económicas o
frustración profesional. La viabilidad y el análisis de riesgos están relacionados
de muchas maneras, si el riesgo del proyecto es alto, la viabilidad de producir
software de calidad se reduce. Así pues, en el presente capítulo aborda los siguientes
puntos:

\begin{itemize}
    \item \textbf{Descripción} formal del proyecto.
    \item \textbf{Objeto}, motivos y necesidades del usuario.
    \item \textbf{Objetivos}, declaración del objetivo final del proyecto.
    \item \textbf{Metodología}, introducción al proceso del ciclo de vida elegido.
    \item \textbf{Requerimientos y recursos necesarios} para alcanzar el objetivo.
    \item \textbf{Sistema de control} para el seguimiento de los hitos principales.
    \item \textbf{Elementos de riesgo} que pueden hacer inviable el proyecto.
    \item \textbf{Beneficios} esperados tras la realización del proyecto.
    \item \textbf{Conclusiones} y resoluciones finales.
\end{itemize}
